Der MD2 Algorithmus ist deterministisch, berechnet also für gleiche Eingabewerte 
gleiche Hashwerte.
Verschiedene Eingabewerte führen mit Ausnahmen von Kollisionen
zu unterschiedlichen Ausgaben.
\\
Da wir hier mit identischen Ausgaben arbeiten, eignet sich eine Diskussion zu 
Genauigkeit in diesem Abschnitt nicht.

\subsection{Testfälle}
Um die Korrektheit unserer Implementierung zu überprüfen, haben wir eine Reihe von Tests
durchgeführt.
Dabei haben wir die von uns berechneten Hashwerte mit denen verglichen, die von verschiedenen
Online-MD2-Rechnern\cite{hashcalculator} erzeugt wurden.
\\
Eine normale .txt-Textdatei wird ausgewählt und der Inhalt dieser Datei
mit unserer Implementierung gehasht.
Anschließend wird dieselbe Datei auf einem Online-MD2-Rechner hochgeladen und
die beiden Hashwerte miteinander verglichen.
Diese Tests sollen sicherstellen, dass unsere Implementierung den Inhalt der
Textdatei nach MD2
korrekt hasht.

\subsection{Fehleranalyse}
Eine vorläufige Implementierung hat nicht zu den gewünschten Ergebnissen geführt.
Unsere Ergebnisse weichten erheblich von den online generierten Ergebnissen ab.
Ferner generierten verschiedene Online-MD2-Rechner unterschiedliche Hashwerte, 
wobei einige mit unseren übereinstimmten.

\subsubsection{Nicht-Text-Dateien}
MD2 behandelt eine rohe Datei als Texteingabe, was das Hashen anderer Dateitypen 
erlauben sollte.
\\
Wir testeten unsere Implementierung mit verschiedenen Dateiformaten wie .o, 
.pdf, .png, .tar usw.
erneut.
Da aber dann der rohe Inhalt der Datei gehasht wird und nicht nur Text, muss der Inhalt der Datei
mit \texttt{uint\_8* buf} anstelle von \texttt{char* buf} dargestellt werden.
Operationen wie \texttt{buf}$[\texttt{len}] = \backslash\texttt{0;}$ können nun 
ausgelassen werden, da der \texttt{string} Datentyp nicht in Verwendung kommt.
\\
Die Leselänge der Quelldatei wird durch die Erstellung der \texttt{stat}-Variablen $statbuf$
und \texttt{fstat(fileno(file), \&statbuf)} ermittelt \cite{fileIO}.
\\
Nach der Verbesserung aller oben beschriebenen Punkte berechnet unser Algorithmus einen
Hash, der in allen Fällen mit dem Wert des Online-MD2-Rechners übereinstimmt.
Dies bestätigt die Korrektheit unserer Implementierung und erlaubt uns die Performanz 
der Implementierung zu testen.
