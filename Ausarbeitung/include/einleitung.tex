Im digitalen Zeitalter sind die Sicherheit und die Integrität von Daten von großer
Bedeutung.
Kryptographische Hashfunktionen spielen eine entscheidende Rolle in
diesem Bereich und sind weitreichend in verschiedenen Domänen eingesetzt.
Der
Fokus dieser Arbeit liegt auf dem MD2 Hashing Algorithmus, welcher im RFC 1319
der Internet Engineering Task Force (IETF) beschrieben ist~\cite{rfc1319}.
\\
Obwohl der MD2 Algorithmus mittlerweile als obsolet gilt, bietet er einen
interessanten Ansatz zur Untersuchung der Grundlagen und der Entwicklung
kryptographischer Hashfunktionen.
\\
Die zu verarbeitenden Daten in solchen Hashing Algorithmen müssen stets ein
Vielfaches der Blocklänge sein, was durch Padding erreicht wird.
Eines der
Verfahren, das diese Anforderung erfüllt, ist das PKCS\#7~\cite{rfc2315}.
\\
Die erste Aufgabe besteht darin, dieses Verfahren genauer zu erforschen und in
dieser Arbeit zu beschreiben.
\\
Darüber hinaus gilt es, die genaue Funktionsweise der Berechnung von Prüfsumme
und Hashwert zu erläutern.
Insbesondere wird in beiden Schritten eine Substitutionsbox verwendet, deren
Werte es zu klären gilt.
\\
Letztendlich werden wir die Angriffe auf den MD2-Algorithmus sowie die Alternativen,
die heute verwendet werden, diskutieren.
Dadurch werden die Gründe für das Auslaufen des MD2 und die Fortschritte in der
Kryptographie verdeutlicht.